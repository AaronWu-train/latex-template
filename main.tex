% !TeX program = xelatex
% !TeX options = -shell-escape
% !TEX encoding = UTF-8

% --------------------------------------------------------------
% This is all preamble stuff that you don't have to worry about.
% Head down to where it says "Start here"
% --------------------------------------------------------------

\documentclass[12pt]{article}
\usepackage[margin=2cm]{geometry} % 頁面邊距設置
\usepackage{amsmath,amssymb,amsthm} % 數學相關套件
\usepackage{graphicx} % 圖形插入
\usepackage{booktabs} % 表格線條
\usepackage[AutoFakeBold,AutoFakeSlant]{xeCJK} % 中文支援
\usepackage{fancyhdr} % 頁首頁尾設置
\usepackage{xcolor,changepage,mdframed} % 顏色與框線相關
\usepackage{titlesec} % 標題格式
\usepackage{setspace} % 行距
\usepackage{minted} % 程式碼高亮(需安裝 Python 套件 pygments)
\usepackage[shortlabels]{enumitem} % 列表格式
\usepackage[colorlinks=true,linkcolor=black,urlcolor=blue,citecolor=blue]{hyperref} % 超連結
\graphicspath{{images}}

% 字體設定 (xeCJK)
\setCJKmainfont{黑體-繁}
\setCJKmonofont{Noto Sans CJK TC}

% 頁首/頁尾設定 (fancyhdr)
\pagestyle{fancy}
\fancyhead[L]{NASA Homework 0} % 可以用 \leftmark 取代
\fancyhead[R]{B13901011 吳亞倫}
\fancyfoot[C]{\thepage}
\renewcommand{\headrulewidth}{0.4pt}
\renewcommand{\footrulewidth}{0pt}
\setlength{\headheight}{15pt}

% tex-fmt: off
\newtheorem*{theorem}{Theorem}
\newenvironment{lemma}[2][Lemma]{\begin{trivlist}
\item[\hskip \labelsep {\bfseries #1}\hskip \labelsep {\bfseries #2.}]}{\end{trivlist}}
\newenvironment{exercise}[2][Exercise]{\begin{trivlist}
\item[\hskip \labelsep {\bfseries #1}\hskip \labelsep {\bfseries #2.}]}{\end{trivlist}}
\newenvironment{problem}[2][Problem]{\begin{trivlist}
\item[\hskip \labelsep {\bfseries #1}\hskip \labelsep {\bfseries #2}]}{\end{trivlist}}
\newenvironment{question}[2][Question]{\begin{trivlist}
\item[\hskip \labelsep {\bfseries #1}\hskip \labelsep {\bfseries #2.}]}{\end{trivlist}}
\newenvironment{corollary}[2][Corollary]{\begin{trivlist}
\item[\hskip \labelsep {\bfseries #1}\hskip \labelsep {\bfseries #2.}]}{\end{trivlist}}
\newenvironment{solution}{\begin{proof}[Solution]}{\end{proof}}
\newenvironment{answer}{\begin{proof}[Ans]}{\end{proof}}

% 樣式設定
% \renewcommand\thesubsection{\arabic{subsection}}
% \titleformat{\subsection}{\normalfont\large\bfseries}{\thesubsection.}{0.5em}{}

% 自訂框線環境
\newmdenv[
  linewidth=2pt, linecolor=gray, backgroundcolor=gray!10,
  topline=false, bottomline=false, rightline=false,
  skipabove=10pt, skipbelow=10pt, innerleftmargin=10pt,
  innerrightmargin=0pt, innertopmargin=5pt, innerbottommargin=5pt
]{myquote}

\newenvironment{mycolorbox}[1][gray]{
  \renewmdenv[
    linewidth=2pt, linecolor=#1, backgroundcolor=#1!10,
    topline=false, bottomline=false, rightline=false,
    skipabove=10pt, skipbelow=10pt, innerleftmargin=10pt,
    innerrightmargin=0pt, innertopmargin=5pt, innerbottommargin=5pt
  ]{myquote}
  \begin{myquote}
}{\end{myquote}}

\linespread{1.1}
% \usemintedstyle{xcode}
\setminted{frame=lines, linenos, framesep=2mm, breaklines=true, breakanywhere=true, autogobble=true}

% tex-fmt: on
\begin{document}
\thispagestyle{fancy}

% --------------------------------------------------------------
%                         Start here
% --------------------------------------------------------------

\begin{center}
  {\Large \bfseries CS 作業標題}
\end{center}

\vspace{2em}

\begin{tabular}{@{}l l}
  \textbf{Student:} & 吳亞倫 \\
  \textbf{ID:} & B13901011  \\
  \textbf{Department:} & 電機工程學系
\end{tabular}
\bigskip

\tableofcontents
\clearpage

\section{簡介}
本模板使用 \texttt{xeCJK} 以支援中文排版;並配置了\href{https://youtu.be/dQw4w9WgXcQ}{連結}、插入圖片、程式高亮功能。

\section{常用語法}
\subsection{Heading 2}
\subsubsection{Heading 3}
\textbf{BOLD FONT}
\textit{italic font}
\emph{emph font}
\begin{enumerate}
  \item 第一點
  \item 第二點
\end{enumerate}
\begin{itemize}
  \item 第一點
  \item 第二點
\end{itemize}
\hrule
\begin{quote}
  This is some quote
\end{quote}
\begin{quotation}
  這裡是一大段引文,比較長時選用quotation環境。
\end{quotation}

\section{插入圖片示範}
\begin{figure}[ht]
  \centering
  \includegraphics[width=0.3\textwidth]{example.png} % 請將 example.png 放在同資料夾
  \caption{示範插入圖片}
  \label{fig:example}
\end{figure}

\section{數學公式示範}
\[
  \int_{0}^{\infty} x^n e^{-x}\, dx = n!
\]

\section{程式碼語法上色示範}
\begin{minted}{python}
def factorial(n):
    if n <= 1:
        return 1
    return n * factorial(n-1)

print(factorial(5))
\end{minted}

\section{看起來是一些酷酷區塊}

\begin{problem}{1}
  Lorem ipsum
\end{problem}

\begin{answer}
  Lorem ipsum dolor sit amet
\end{answer}

\begin{exercise}{123}
  Lorem ipsum
  \begin{equation*}
    \int_0^{3\pi/2}|\sin x|\,dx
  \end{equation*}
\end{exercise}

\begin{solution}
  Lorem ipsum dolor sit amet
  \begin{equation*}
    \begin{split}
      &  \int_0^{3\pi/2}|\sin x|\,dx \\
    \end{split}
  \end{equation*}
\end{solution}

% --------------------------------------------------------------
%     You don't have to mess with anything below this line.
% --------------------------------------------------------------

\end{document}
