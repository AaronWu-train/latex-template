% !TeX program = xelatex
% !TeX options = -shell-escape
% !TEX encoding = UTF-8

% --------------------------------------------------------------
% This is all preamble stuff that you don't have to worry about.
% Head down to where it says "Start here"
% --------------------------------------------------------------

\documentclass[12pt]{article}
\usepackage[margin=1in]{geometry}
\usepackage{amsmath,amsthm,amssymb}
\usepackage{graphicx}
\usepackage[colorlinks=true,linkcolor=black,urlcolor=blue,citecolor=blue]{hyperref}
\usepackage{minted} % Code highlighting, required python and pygments
\usepackage{titlesec}
\usepackage[shortlabels]{enumitem}

% 中文支援 (xeCJK)
\usepackage[AutoFakeBold,AutoFakeSlant]{xeCJK}
\setCJKmainfont{源流明體}
\setCJKmonofont{Noto Sans CJK TC}

% 自訂頁首/頁尾 (fancyhdr)
\usepackage{fancyhdr}
\pagestyle{fancy}
\fancyhead[L]{\leftmark}			  % 左上,可用 \leftmark 顯示章節標題
\fancyhead[R]{B13901011 吳亞倫} 
\fancyfoot[C]{\thepage} 
\renewcommand{\headrulewidth}{0.4pt}  % 頁首下方的線條寬度
\renewcommand{\footrulewidth}{0pt}    % 頁尾上方的線條寬度 (不顯示)
\setlength{\headheight}{15pt}

\newtheorem*{theorem}{Theorem}
\newenvironment{lemma}[2][Lemma]{\begin{trivlist}
\item[\hskip \labelsep {\bfseries #1}\hskip \labelsep {\bfseries #2.}]}{\end{trivlist}}
\newenvironment{exercise}[2][Exercise]{\begin{trivlist}
\item[\hskip \labelsep {\bfseries #1}\hskip \labelsep {\bfseries #2.}]}{\end{trivlist}}
\newenvironment{problem}[2][Problem]{\begin{trivlist}
\item[\hskip \labelsep {\bfseries #1}\hskip \labelsep {\bfseries #2.}]}{\end{trivlist}}
\newenvironment{question}[2][Question]{\begin{trivlist}
\item[\hskip \labelsep {\bfseries #1}\hskip \labelsep {\bfseries #2.}]}{\end{trivlist}}
\newenvironment{corollary}[2][Corollary]{\begin{trivlist}
\item[\hskip \labelsep {\bfseries #1}\hskip \labelsep {\bfseries #2.}]}{\end{trivlist}}
\newenvironment{solution}{\begin{proof}[Solution]}{\end{proof}}
\newenvironment{answer}{\begin{proof}[Answer]}{\end{proof}}
% \titleformat{\section}{\bfseries\Large}{Section }{0pt}{}

\begin{document}
\thispagestyle{fancy} 

% --------------------------------------------------------------
%                         Start here
% --------------------------------------------------------------

\begin{center}
    {\Large \bfseries CS 作業標題}
\end{center}

\vspace{2em}

\begin{tabular}{@{}l l}
\textbf{Student:} & 吳亞倫 \\
\textbf{ID:} & B13901011  \\
\textbf{Department:} & 電機工程學系
\end{tabular}
\bigskip

\tableofcontents
\clearpage

\section{簡介}
本模板使用 \texttt{xeCJK} 以支援中文排版;並配置了\href{https://youtu.be/dQw4w9WgXcQ}{連結}、插入圖片、程式高亮功能。

\section{常用語法}
\subsection{Heading 2}
\subsubsection{Heading 3}
\textbf{BOLD FONT}
\textit{italic font}
\emph{emph font}
\begin{enumerate}
  \item 第一點
  \item 第二點
\end{enumerate}
\begin{itemize}
  \item 第一點
  \item 第二點
\end{itemize}
\hrule
\begin{quote}
This is some quote
\end{quote}
\begin{quotation}
這裡是一大段引文,比較長時選用quotation環境。
\end{quotation}


\section{插入圖片示範}
\begin{figure}[ht]
    \centering
    \includegraphics[width=0.3\textwidth]{example.png} % 請將 example.png 放在同資料夾
    \caption{示範插入圖片}
    \label{fig:example}
\end{figure}

\section{數學公式示範}
\[
    \int_{0}^{\infty} x^n e^{-x}\, dx = n!
\]

\section{程式碼語法上色示範}
\begin{minted}[frame=lines, linenos]{python}
def factorial(n):
    if n <= 1:
        return 1
    return n * factorial(n-1)

print(factorial(5))
\end{minted}

\section{看起來是一些酷酷區塊}

\begin{problem}{1}
    Lorem ipsum
\end{problem}

\begin{solution}
    Lorem ipsum dolor sit amet
\end{solution}

\begin{exercise}{123}
    Lorem ipsum
    \begin{equation*}
        \int_0^{3\pi/2}|\sin x|\,dx
    \end{equation*}
\end{exercise}

\begin{answer}
    Lorem ipsum dolor sit amet
    \begin{equation*}
        \begin{split}
            &  \int_0^{3\pi/2}|\sin x|\,dx \\
        \end{split}
    \end{equation*}
\end{answer}

% --------------------------------------------------------------
%     You don't have to mess with anything below this line.
% --------------------------------------------------------------

\end{document}

